%!TEX program = xelatex
% Encoding: UTF8
% SEIKA 2015

%\documentclass[a4paper,11pt,twoside]{book}
\documentclass[a4paper,11pt,twoside]{ctexbook}

\usepackage{geometry}
\geometry{left=3.5cm, right=3cm, top=3cm, bottom=3cm}
%控制页眉页脚页码
\pagestyle{headings}
%罗马字符页码
%\pagenumbering{roman}

% \usepackage{ctex}
% \usepackage{xeCJK}

%\CJKsetecglue{} % 禁用汉字与其他内容之间空格(空隙)

% 支持西文字体
\usepackage{fourier}
\usepackage{courier}
% \usepackage{fontspec}
\newfontfamily\CodeFont{Ubuntu Mono}

\usepackage{graphicx}
% 支持插入eps图形文件
% \usepackage{epsfig}

% 支持超链接
\usepackage[colorlinks]{hyperref}

% 支持代码框插入
\usepackage{xcolor}
\definecolor{mygreen}{rgb}{0,0.6,0}
\definecolor{mygray}{rgb}{0.5,0.5,0.5}
\definecolor{mymauve}{rgb}{0.58,0,0.82}
\definecolor{myback}{rgb}{0.95,0.92,0.93}

\usepackage{listings}
\lstset{ %
  backgroundcolor=\color{myback},    % choose the background color; you must add \usepackage{color} or \usepackage{xcolor}
  basicstyle=\linespread{0.95}\small\CodeFont,   % the size of the fonts that are used for the code
  breakatwhitespace=false,           % sets if automatic breaks should only happen at whitespace
  breaklines=true,                   % sets automatic line breaking
  captionpos=bl,                     % sets the caption-position to bottom
  commentstyle=\color{mygreen},      % comment style
  deletekeywords={...},              % if you want to delete keywords from the given language
  escapeinside={\%*}{*)},            % if you want to add LaTeX within your code
  extendedchars=true,                % lets you use non-ASCII characters; for 8-bits encodings only, does not work with UTF-8
  frame=single,                      % adds a frame around the code
  frameround=tttt,
  keepspaces=true,                   % keeps spaces in text, useful for keeping indentation of code (possibly needs columns=flexible)
  keywordstyle=\color{blue},         % keyword style
  language=Python,                   % the language of the code
  morekeywords={*,...},              % if you want to add more keywords to the set
  numbers=left,                      % where to put the line-numbers; possible values are (none, left, right)
  numbersep=4pt,                     % how far the line-numbers are from the code
  numberstyle=\tiny\CodeFont\color{mygray},   % the style that is used for the line-numbers
  rulecolor=\color{mygray},          % if not set, the frame-color may be changed on line-breaks within not-black text (e.g. comments (green here))
  showspaces=false,                  % show spaces everywhere adding particular underscores; it overrides 'showstringspaces'
  showstringspaces=true,             % underline spaces within strings only
  showtabs=true,                     % show tabs within strings adding particular underscores
  stepnumber=1,                      % the step between two line-numbers. If it's 1, each line will be numbered
  stringstyle=\color{orange},        % string literal style
  tabsize=2,                         % sets default tabsize to 2 spaces
  %title=myPython.py                 % show the filename of files included with \lstinputlisting; also try caption instead of title
  xleftmargin = 2em,
  xrightmargin = 2em,
  aboveskip = 0.5 em
}

% \setCJKmainfont[BoldFont={SimSun},ItalicFont={KaiTi}] %{SimSun}

%%%%%%%%%%%%
\title{TensorFlow 指南}
\author{}
\date{\today}
% \thanks{}

\begin{document}

\maketitle
\tableofcontents

%%%% 第一章
\newpage
\chapter{起步}
% \section{Introduction}
\include{get_started/sec_1_1_introduction}
\include{get_started/sec_1_2_os_setup}
\include{get_started/sec_1_3_basic_usage}

%%%% 第二章
\newpage
\chapter{基础教程}
%!TEX program = xelatex
% Encoding: UTF8
% SEIKA 2015


% Chapter 2 TutorialsHow to ...
% Section 2.1

\chapter{基础教程}

\section{综述}

\textbf{MNIST For ML Beginners}

If you're new to machine learning, we recommend starting here. You'll learn about a classic problem, handwritten digit classification (MNIST), and get a gentle introduction to multiclass classification.

View Tutorial

\textbf{Deep MNIST for Experts}

If you're already familiar with other deep learning software packages, and are already familiar with MNIST, this tutorial with give you a very brief primer on TensorFlow.

View Tutorial

\textbf{TensorFlow Mechanics 101}

This is a technical tutorial, where we walk you through the details of using TensorFlow infrastructure to train models at scale. We use again MNIST as the example.

View Tutorial

\textbf{Convolutional Neural Networks}

An introduction to convolutional neural networks using the CIFAR-10 data set. Convolutional neural nets are particularly tailored to images, since they exploit translation invariance to yield more compact and effective representations of visual content.

View Tutorial

\textbf{Vector Representations of Words}

This tutorial motivates why it is useful to learn to represent words as vectors (called word embeddings). It introduces the word2vec model as an efficient method for learning embeddings. It also covers the high-level details behind noise-contrastive training methods (the biggest recent advance in training embeddings).

View Tutorial

\textbf{Recurrent Neural Networks}

An introduction to RNNs, wherein we train an LSTM network to predict the next word in an English sentence. (A task sometimes called language modeling.)

View Tutorial

\textbf{Sequence-to-Sequence Models}

A follow on to the RNN tutorial, where we assemble a sequence-to-sequence model for machine translation. You will learn to build your own English-to-French translator, entirely machine learned, end-to-end.

View Tutorial

\textbf{Mandelbrot Set}

TensorFlow can be used for computation that has nothing to do with machine learning. Here's a naive implementation of Mandelbrot set visualization.

View Tutorial

\textbf{Partial Differential Equations}

As another example of non-machine learning computation, we offer an example of a naive PDE simulation of raindrops landing on a pond.

View Tutorial

\textbf{MNIST Data Download}

Details about downloading the MNIST handwritten digits data set. Exciting stuff.

View Tutorial

\textbf{Image Recognition}

How to run object recognition using a convolutional neural network trained on ImageNet Challenge data and label set.

View Tutorial

We will soon be releasing code for training a state-of-the-art Inception model.

Deep Dream Visual Hallucinations

Building on the Inception recognition model, we will release a TensorFlow version of the Deep Dream neural network visual hallucination software.

COMING SOON
\include{tutorials/sec_2_2_minist_beginners}
\include{tutorials/sec_2_3_minist_pros}
\include{tutorials/sec_2_4_}
%!TEX program = xelatex
% Encoding: UTF8
% SEIKA 2015


% Chapter 2 Tutorials
% Section 2.5


\newpage
\section {卷积神经网络} \label{cnn}

\subsection {Overview \footnote{This tutorial is intended for advanced users of TensorFlow and assumes expertise and experience in machine learning}}

CIFAR-10 classification is a common benchmark problem in machine learning. The problem is to classify RGB 32x32 pixel images across 10 categories: airplane, automobile, bird, cat, deer, dog, frog, horse, ship, and truck.

对CIFAR-10 数据集的分类是机器学习中一个公开的基准测试问题,其任务是对一组32x32RGB的图像进行分类,这些图像涵盖了10个类别:\lstinline{airplane}, \lstinline{automobile}, \lstinline{bird}, \lstinline{cat}, \lstinline{deer}, \lstinline{dog}, \lstinline{frog}, \lstinline{horse}, \lstinline{ship}, 和 \lstinline{truck}.

\begin{center}
\includegraphics[width=.55\textwidth]{../SOURCE/images/cifar_samples.png}
\end{center}

想了解更多信息请参考\href{http://www.cs.toronto.edu/~kriz/cifar.html}{CIFAR-10 page},以及Alex Krizhevsky写的\href{http://www.cs.toronto.edu/~kriz/learning-features-2009-TR.pdf}{技术报告}。

\subsubsection {G目标}
本教程的目标是建立一个用于识别图像的相对较小的卷积神经网络,在这一过程中,本教程会:

\begin{itemize}
\item 着重于建立一个规范的网络组织结构,训练并进行评估;
\item 为建立更大规模更加复杂的模型提供一个范例
\end{itemize}

选择CIFAR-10是因为它的复杂程度足以用来检验TensorFlow中的大部分功能,并可将其扩展为更大的模型。与此同时由于模型较小所以训练速度很快,比较适合用来测试新的想法,检验新的技术。

\subsubsection {本教程的重点}
CIFAR-10 教程演示了在TensorFlow上构建更大更复杂模型的几个种重要内容:

\begin{itemize}
\item 相关核心数学对象,如卷积、修正线性激活、最大池化以及局部响应归一化;
\item 训练过程中一些网络行为的可视化,这些行为包括输入图像、损失情况、网络行为的分布情况以及梯度;
\item 算法学习参数的移动平均值的计算函数,以及在评估阶段使用这些平均值提高预测性能;
\item 实现了一种机制,使得学习率随着时间的推移而递减;
\item 为输入数据设计预存取队列,将磁盘延迟和高开销的图像预处理操作与模型分离开来处理;
\end{itemize}

我们也提供了模型的多GUP版本,用以表明:

\begin{itemize}
\item 可以配置模型后使其在多个GPU上并行的训练
\item 可以在多个GPU之间共享和更新变量值
\end{itemize}

我们希望本教程给大家开了个头,使得在Tensorflow上可以为视觉相关工作建立更大型的Cnns模型

\subsubsection {模型架构}

本教程中的模型是一个多层架构,由卷积层和非线性层(nonlinearities)交替多次排列后构成。这些层最终通过全连通层对接到softmax分类器上。这一模型除了最顶部的几层外,基本跟Alex Krizhevsky提出的模型一致。

在一个GPU上经过几个小时的训练后,该模型达到了最高86\%的精度。细节请查看下面的描述以及代码。模型中包含了1,068,298个学习参数,分类一副图像需要大概19.5M个乘加操作。

\subsection {Code Organization}

本教程的代码位于\href{https://tensorflow.googlesource.com/tensorflow/+/master/tensorflow/models/image/cifar10/}{tensorflow/models/image/cifar10/}.

% insert table here

\subsection {CIFAR-10 模型}


\subsubsection {Model Inputs}

\subsubsection {Model Prediction}

\subsubsection {Model Training}

\subsection {Lauching and Training the Model}

\subsection {Evaluating a Model}

\subsection {Traning a Model Using Multiple GPU Cards}

\subsubsection {Placing Variables and Operations on Devices}

\subsubsection {Lauching and Training the Model on Multiple GPU cards}

\subsection {Next Steps}






\newpage
% Chapter 3 How to...
% 第三章 运作方式
%!TEX program = xelatex
% Encoding: UTF8
% SEIKA 2015


% Chapter 3 How to ...
% Section 3.1

\chapter{运作方式}

\section{变量:创建、初始化、保存和加载}

当训练模型时,用变量来存储和更新参数。变量包含张量 (Tensor)存放于内存的缓存区。建模时它们需要被明确地初始化,模型训练后它们必须被存储到磁盘。这些变量的值可在之后模型训练和分析是被加载。

本文档描述以下两个TensorFlow类。点击以下链接可查看完整的API文档:
\begin{itemize}
  \item tf.Variable 类 % add link here
  \item tf.train.Saver 类 % add link here
\end{itemize}

\subsection {创建}

当创建一个变量时,你将一个张量作为初始值传入构造函数Variable()。TensorFlow提供了一系列操作符来初始化张量,初始值是常量或是随机值。
% add link here

\begin{lstlisting}
# Create two variables.
weights = tf.Variable(tf.random_normal([784, 200], stddev=0.35), name="weights")
biases = tf.Variable(tf.zeros([200]), name="biases")
\end{lstlisting}

调用tf.Variable()添加一些操作(Op, operation)到graph:
\begin{itemize}
  \item 一个Variable操作存放变量的值。
  \item 一个初始化op将变量设置为初始值。这事实上是一个tf.assign操作。
  \item 初始值的操作,例如示例中对biases变量的zeros操作也被加入了graph。
\end{itemize}
tf.Variable的返回值是Python的tf.Variable类的一个实例。

\subsection {初始化}

变量的初始化必须在模型的其它操作运行之前先明确地完成。最简单的方法就是添加一个给所有变量初始化的操作,并在使用模型之前首先运行那个操作。

你或者可以从检查点文件中重新获取变量值,详见下文。

使用tf.initialize\_all\_variables()添加一个操作对变量做初始化。记得在完全构建好模型并加载之后再运行那个操作。


\newpage
\chapter{资源}

\newpage
\chapter{其他}

\end{document}