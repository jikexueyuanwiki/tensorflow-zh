%!TEX program = xelatex
% Encoding: UTF8
% SEIKA 2015

%\documentclass[a4paper,11pt,twoside]{book}
\documentclass[a4paper,11pt,twoside]{ctexbook}

\usepackage{geometry}
\geometry{left=3.5cm, right=3cm, top=3cm, bottom=3cm}
%控制页眉页脚页码
\pagestyle{headings}
%罗马字符页码
%\pagenumbering{roman}

%\usepackage{ctex}    uncomment this if article is applied on docclass
%\usepackage{xeCJK}   uncomment this if article is applied on docclass

%\CJKsetecglue{} % 禁用汉字与其他内容之间空格(空隙)
%\usepackage{ctex}

% 支持西文字体
\usepackage{fourier}
\usepackage{courier}
% \usepackage{fontspec}
% \setmonofont{Verdana}

\usepackage{graphicx}
% 支持插入eps图形文件
% \usepackage{epsfig}

% 支持超链接
\usepackage[colorlinks]{hyperref}

% 支持代码框插入
\usepackage{xcolor}
\definecolor{mygreen}{rgb}{0,0.6,0}
\definecolor{mygray}{rgb}{0.5,0.5,0.5}
\definecolor{mymauve}{rgb}{0.58,0,0.82}
\definecolor{myback}{rgb}{0.95,0.95,0.95}

\usepackage{listings}
\lstset{ %
  backgroundcolor=\color{myback},    % choose the background color; you must add \usepackage{color} or \usepackage{xcolor}
  basicstyle=\scriptsize\ttfamily,   % the size of the fonts that are used for the code
  breakatwhitespace=false,           % sets if automatic breaks should only happen at whitespace
  breaklines=true,                   % sets automatic line breaking
  captionpos=bl,                     % sets the caption-position to bottom
  commentstyle=\color{mygreen},      % comment style
  deletekeywords={...},              % if you want to delete keywords from the given language
  escapeinside={\%*}{*)},            % if you want to add LaTeX within your code
  extendedchars=true,                % lets you use non-ASCII characters; for 8-bits encodings only, does not work with UTF-8
  frame=single,                      % adds a frame around the code
  keepspaces=true,                   % keeps spaces in text, useful for keeping indentation of code (possibly needs columns=flexible)
  keywordstyle=\color{blue},         % keyword style
  language=Python,                   % the language of the code
  morekeywords={*,...},              % if you want to add more keywords to the set
  numbers=left,                      % where to put the line-numbers; possible values are (none, left, right)
  numbersep=4pt,                     % how far the line-numbers are from the code
  numberstyle=\tiny\color{mygray},   % the style that is used for the line-numbers
  rulecolor=\color{mygray},          % if not set, the frame-color may be changed on line-breaks within not-black text (e.g. comments (green here))
  showspaces=false,                  % show spaces everywhere adding particular underscores; it overrides 'showstringspaces'
  showstringspaces=true,             % underline spaces within strings only
  showtabs=true,                     % show tabs within strings adding particular underscores
  stepnumber=1,                      % the step between two line-numbers. If it's 1, each line will be numbered
  stringstyle=\color{orange},        % string literal style
  tabsize=2,                         % sets default tabsize to 2 spaces
  %title=myPython.py                 % show the filename of files included with \lstinputlisting; also try caption instead of title
}

% \setCJKmainfont[BoldFont={SimSun},ItalicFont={KaiTi}] %{SimSun}

%%%%%%%%%%%%
\title{Tensorflow 指南}
\author{}
\date{2015-12-16}
% \thanks{}

\begin{document}

\maketitle
\newpage
\tableofcontents
\newpage

\chapter{起步}
\section{Introduction}
\section{Download and Setip}
你可以他通过二进制安装包或者github源安装TensorFlow
\subsection {安装需求}

\subsection {Overview}

TensorFlow Python API 依赖 Python 2.7 版本.

在 Linux 和 Mac 下最简单的安装方式, 是使用 \framebox{pip}安装.

如果在安装过程中遇到错误, 请查阅\framebox{常见问题}. 为了简化安装步骤, 建议使用 virtualenv, 教程见 \verb{这里}.

\newpage
%\chapter{基础教程}

%!TEX program = xelatex
% Encoding: UTF8
% SEIKA 2015


% Chapter 2 TutorialsHow to ...
% Section 2.1

\chapter{基础教程}

\section{综述}

\textbf{MNIST For ML Beginners}

If you're new to machine learning, we recommend starting here. You'll learn about a classic problem, handwritten digit classification (MNIST), and get a gentle introduction to multiclass classification.

View Tutorial

\textbf{Deep MNIST for Experts}

If you're already familiar with other deep learning software packages, and are already familiar with MNIST, this tutorial with give you a very brief primer on TensorFlow.

View Tutorial

\textbf{TensorFlow Mechanics 101}

This is a technical tutorial, where we walk you through the details of using TensorFlow infrastructure to train models at scale. We use again MNIST as the example.

View Tutorial

\textbf{Convolutional Neural Networks}

An introduction to convolutional neural networks using the CIFAR-10 data set. Convolutional neural nets are particularly tailored to images, since they exploit translation invariance to yield more compact and effective representations of visual content.

View Tutorial

\textbf{Vector Representations of Words}

This tutorial motivates why it is useful to learn to represent words as vectors (called word embeddings). It introduces the word2vec model as an efficient method for learning embeddings. It also covers the high-level details behind noise-contrastive training methods (the biggest recent advance in training embeddings).

View Tutorial

\textbf{Recurrent Neural Networks}

An introduction to RNNs, wherein we train an LSTM network to predict the next word in an English sentence. (A task sometimes called language modeling.)

View Tutorial

\textbf{Sequence-to-Sequence Models}

A follow on to the RNN tutorial, where we assemble a sequence-to-sequence model for machine translation. You will learn to build your own English-to-French translator, entirely machine learned, end-to-end.

View Tutorial

\textbf{Mandelbrot Set}

TensorFlow can be used for computation that has nothing to do with machine learning. Here's a naive implementation of Mandelbrot set visualization.

View Tutorial

\textbf{Partial Differential Equations}

As another example of non-machine learning computation, we offer an example of a naive PDE simulation of raindrops landing on a pond.

View Tutorial

\textbf{MNIST Data Download}

Details about downloading the MNIST handwritten digits data set. Exciting stuff.

View Tutorial

\textbf{Image Recognition}

How to run object recognition using a convolutional neural network trained on ImageNet Challenge data and label set.

View Tutorial

We will soon be releasing code for training a state-of-the-art Inception model.

Deep Dream Visual Hallucinations

Building on the Inception recognition model, we will release a TensorFlow version of the Deep Dream neural network visual hallucination software.

COMING SOON
\include{tutorials/sec_2_2_minist_beginners}
\include{tutroials/sec_2_3_minist_pros}

\newpage
\section {TensorFlow运作方式}
\section {卷积神经网络}
\subsection {概述}
\subsection {代码组织}
\subsection {CIFAR-10模型}
\subsection {开始执行并训练模型}
\subsection {模型评估}
\section {Vector Representations of Words}
\section {循环神经网络}
\section {曼德博(Mandelbrot)集合}
\section {偏微分方程}
\section {MNIST数据集下载}


\newpage


% Chapter 3 How to...
% 第三章 运作方式
%!TEX program = xelatex
% Encoding: UTF8
% SEIKA 2015


% Chapter 3 How to ...
% Section 3.1

\chapter{运作方式}

\section{变量:创建、初始化、保存和加载}

当训练模型时,用变量来存储和更新参数。变量包含张量 (Tensor)存放于内存的缓存区。建模时它们需要被明确地初始化,模型训练后它们必须被存储到磁盘。这些变量的值可在之后模型训练和分析是被加载。

本文档描述以下两个TensorFlow类。点击以下链接可查看完整的API文档:
\begin{itemize}
  \item tf.Variable 类 % add link here
  \item tf.train.Saver 类 % add link here
\end{itemize}

\subsection {创建}

当创建一个变量时,你将一个张量作为初始值传入构造函数Variable()。TensorFlow提供了一系列操作符来初始化张量,初始值是常量或是随机值。
% add link here

\begin{lstlisting}
# Create two variables.
weights = tf.Variable(tf.random_normal([784, 200], stddev=0.35), name="weights")
biases = tf.Variable(tf.zeros([200]), name="biases")
\end{lstlisting}

调用tf.Variable()添加一些操作(Op, operation)到graph:
\begin{itemize}
  \item 一个Variable操作存放变量的值。
  \item 一个初始化op将变量设置为初始值。这事实上是一个tf.assign操作。
  \item 初始值的操作,例如示例中对biases变量的zeros操作也被加入了graph。
\end{itemize}
tf.Variable的返回值是Python的tf.Variable类的一个实例。

\subsection {初始化}

变量的初始化必须在模型的其它操作运行之前先明确地完成。最简单的方法就是添加一个给所有变量初始化的操作,并在使用模型之前首先运行那个操作。

你或者可以从检查点文件中重新获取变量值,详见下文。

使用tf.initialize\_all\_variables()添加一个操作对变量做初始化。记得在完全构建好模型并加载之后再运行那个操作。
% \input{how_tos/sec_1_variables}


\newpage
\chapter{资源}

\newpage
\chapter{其他}

\end{document}