%!TEX program = xelatex
% Encoding: UTF8
% SEIKA 2015


% Chapter 3 How to ...
% Section 3.1



\section{变量:创建、初始化、保存和加载}\label{variables}

当训练模型时,用变量来存储和更新参数。变量包含张量 (Tensor)存放于内存的缓存区。建模时它们需要被明确地初始化,模型训练后它们必须被存储到磁盘。这些变量的值可在之后模型训练和分析是被加载。

本文档描述以下两个TensorFlow类。点击以下链接可查看完整的API文档:
\begin{itemize}
  \item tf.Variable 类 % add link here
  \item tf.train.Saver 类 % add link here
\end{itemize}

\subsection {变量创建}

当创建一个变量时,你将一个张量作为初始值传入构造函数Variable()。TensorFlow提供了一系列操作符来初始化张量,初始值是常量或是随机值。
% add link here

\begin{lstlisting}
# Create two variables.
weights = tf.Variable(tf.random_normal([784, 200], stddev=0.35), name="weights")
biases = tf.Variable(tf.zeros([200]), name="biases")
\end{lstlisting}

调用tf.Variable()添加一些操作(Op, operation)到graph:
\begin{itemize}
  \item 一个Variable操作存放变量的值。
  \item 一个初始化op将变量设置为初始值。这事实上是一个tf.assign操作。
  \item 初始值的操作,例如示例中对biases变量的zeros操作也被加入了graph。
\end{itemize}
\lstinline{tf.Variable}的返回值是Python的\lstinline{tf.Variable}类的一个实例。

\subsection {变量初始化}

变量的初始化必须在模型的其它操作运行之前先明确地完成。最简单的方法就是添加一个给所有变量初始化的操作,并在使用模型之前首先运行那个操作。

你或者可以从检查点文件中重新获取变量值,详见下文。

使用\lstinline{tf.initialize_all_variables()}添加一个操作对变量做初始化。记得在完全构建好模型并加载之后再运行那个操作。

\begin{lstlisting}
# Create two variables.
weights = tf.Variable(tf.random_normal([784, 200], stddev=0.35),
                      name="weights")
biases = tf.Variable(tf.zeros([200]), name="biases")
...
# Add an op to initialize the variables.
init_op = tf.initialize_all_variables()

# Later, when launching the model
with tf.Session() as sess:
  # Run the init operation.
  sess.run(init_op)
  ...
  # Use the model
  ...
\end{lstlisting}

\subsubsection{由另一个变量初始化}
你有时候会需要用另一个变量的初始化值给当前变量初始化。由于\lstinline{tf.initialize_all_variables()}是并行地初始化所有变量,所以在有这种需求的情况下需要小心。

用其它变量的值初始化一个新的变量时,使用其它变量的\lstinline{initialized_value()}属性。你可以直接把已初始化的值作为新变量的初始值,或者把它当做tensor计算得到一个值赋予新变量。

\begin{lstlisting}
# Create a variable with a random value.
weights = tf.Variable(tf.random_normal([784, 200], stddev=0.35),name="weights")
# Create another variable with the same value as 'weights'.
w2 = tf.Variable(weights.initialized_value(), name="w2")
# Create another variable with twice the value of 'weights'
w_twice = tf.Variable(weights.initialized_value() * 0.2, name="w_twice")
\end{lstlisting}

\subsubsection{自定义初始化}

\lstinline{tf.initialize_all_variables()}函数便捷地添加一个op来初始化模型的所有变量。你也可以给它传入一组变量进行初始化。详情请见Variables Documentation,包括检查变量是否被初始化。

% add link of Variables Documentaion here
\subsection {保存和加载}
最简单的保存和恢复模型的方法是使用\lseinline{tf.train.Saver}对象。构造器给graph的所有变量,或是定义在列表里的变量,添加\lstinline{save}和\lstinline{restoreops}。\lstinline{saver}对象提供了方法来运行这些ops,定义检查点文件的读写路径。

\subsubsection{Checkpoint Files}
Variables are saved in binary files that, roughly, contain a map from variable names to tensor values.

When you create a Saver object, you can optionally choose names for the variables in the checkpoint files. By default, it uses the value of the \lstinline{Variable.name} property for each variable.
% add link of Variable.name here
% https://www.tensorflow.org/versions/master/api_docs/python/state_ops.html#Variable.name

\subsubsection{保存变量}

用\lstinline{tf.train.Saver()}创建一个\lstinline{Saver}来管理模型中的所有变量。

\begin{lstlisting}
# Create some variables.
v1 = tf.Variable(..., name="v1")
v2 = tf.Variable(..., name="v2")
...
# Add an op to initialize the variables.
init_op = tf.initialize_all_variables()

# Add ops to save and restore all the variables.
saver = tf.train.Saver()

# Later, launch the model, initialize the variables, do some work, save the
# variables to disk.
with tf.Session() as sess:
  sess.run(init_op)
  # Do some work with the model.
  ..
  # Save the variables to disk.
  save_path = saver.save(sess, "/tmp/model.ckpt")
  print "Model saved in file: ", save_path
\end{lstlisting}

\subsubsection{恢复变量}

用同一个\lstinline{Saver}对象来恢复变量。注意,当你从文件中恢复变量时,不需要事先对它们做初始化。

\begin{lstlisting}
# Create some variables.
v1 = tf.Variable(..., name="v1")
v2 = tf.Variable(..., name="v2")
...
# Add ops to save and restore all the variables.
saver = tf.train.Saver()

# Later, launch the model, use the saver to restore variables from disk, and
# do some work with the model.
with tf.Session() as sess:
  # Restore variables from disk.
  saver.restore(sess, "/tmp/model.ckpt")
  print "Model restored."
  # Do some work with the model
  ...
\end{lstlisting}

\subsubsection{选择存储和恢复哪些变量}
如果你不给\lstinline{tf.train.Saver()}传入任何参数,那么\lstinline{saver}将处理\lstinline{graph}中的所有变量。其中每一个变量都以变量创建时传入的名称被保存。

有时候在检查点文件中明确定义变量的名称很有用。举个例子,你也许已经训练得到了一个模型,其中有个变量命名为\lstinline{"weights"},你想把它的值恢复到一个新的变量\lstinline{"params"}中。

有时候仅保存和恢复模型的一部分变量很有用。再举个例子,你也许训练得到了一个5层神经网络,现在想训练一个6层的新模型,可以将之前5层模型的参数导入到新模型的前5层中。

你可以通过给\lstinline{tf.train.Saver()}构造函数传入Python字典,很容易地定义需要保持的变量及对应名称:键对应使用的名称,值对应被管理的变量。

\textbf{注意}:

\begin{quote}
You can create as many saver objects as you want if you need to save and restore different subsets of the model variables. The same variable can be listed in multiple saver objects, its value is only changed when the saver restore() method is run.

If you only restore a subset of the model variables at the start of a session, you have to run an initialize op for the other variables. See \lstinline{tf.initialize_variables()} for more information.
\end{quote}

如果需要保存和恢复模型变量的不同子集,可以创建任意多个saver对象。同一个变量可被列入多个saver对象中,只有当saver的\lstinline{restore()}函数被运行时,它的值才会发生改变。

如果你仅在session开始时恢复模型变量的一个子集,你需要对剩下的变量执行初始化op。详情请见\lstinline{tf.initialize_variables()}。

% add link of tf.initialize_variables() here
% https://www.tensorflow.org/versions/master/api_docs/python/state_ops.html#initialize_variables

\begin{lstlisting}
# Create some variables.
v1 = tf.Variable(..., name="v1")
v2 = tf.Variable(..., name="v2")
...
# Add ops to save and restore only 'v2' using the name "my_v2"
saver = tf.train.Saver({"my_v2": v2})
# Use the saver object normally after that.
...
\end{lstlisting}