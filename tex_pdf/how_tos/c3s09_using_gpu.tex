

\section{使用 GPUs }\label{ux4f7fux7528-gpus}

\subsection{支持的设备 }\label{ux652fux6301ux7684ux8bbeux5907}

在一套标准的系统上通常有多个计算设备. TensorFlow 支持 CPU 和 GPU
这两种设备. 我们用指定字符串 \texttt{strings} 来标识这些设备. 比如:

\begin{itemize}
\tightlist
\item
  \texttt{"/cpu:0"}: 机器中的 CPU
\item
  \texttt{"/gpu:0"}: 机器中的 GPU, 如果你有一个的话.
\item
  \texttt{"/gpu:1"}: 机器中的第二个 GPU, 以此类推\ldots{}
\end{itemize}

如果一个 TensorFlow 的 operation 中兼有 CPU 和 GPU 的实现,
当这个算子被指派设备时, GPU 有优先权. 比如\texttt{matmul}中 CPU 和 GPU
kernel 函数都存在. 那么在 \texttt{cpu:0} 和 \texttt{gpu:0} 中,
\texttt{matmul} operation 会被指派给 \texttt{gpu:0} .

\subsection{记录设备指派情况
}\label{ux8bb0ux5f55ux8bbeux5907ux6307ux6d3eux60c5ux51b5}

为了获取你的 operations 和 Tensor 被指派到哪个设备上运行, 用
\texttt{log\_device\_placement} 新建一个 \texttt{session}, 并设置为
\texttt{True}.

\begin{Shaded}
\begin{Highlighting}[]
\CommentTok{# 新建一个 graph.}
\NormalTok{a }\OperatorTok{=} \NormalTok{tf.constant([}\FloatTok{1.0}\NormalTok{, }\FloatTok{2.0}\NormalTok{, }\FloatTok{3.0}\NormalTok{, }\FloatTok{4.0}\NormalTok{, }\FloatTok{5.0}\NormalTok{, }\FloatTok{6.0}\NormalTok{], shape}\OperatorTok{=}\NormalTok{[}\DecValTok{2}\NormalTok{, }\DecValTok{3}\NormalTok{], name}\OperatorTok{=}\StringTok{'a'}\NormalTok{)}
\NormalTok{b }\OperatorTok{=} \NormalTok{tf.constant([}\FloatTok{1.0}\NormalTok{, }\FloatTok{2.0}\NormalTok{, }\FloatTok{3.0}\NormalTok{, }\FloatTok{4.0}\NormalTok{, }\FloatTok{5.0}\NormalTok{, }\FloatTok{6.0}\NormalTok{], shape}\OperatorTok{=}\NormalTok{[}\DecValTok{3}\NormalTok{, }\DecValTok{2}\NormalTok{], name}\OperatorTok{=}\StringTok{'b'}\NormalTok{)}
\NormalTok{c }\OperatorTok{=} \NormalTok{tf.matmul(a, b)}
\CommentTok{# 新建session with log_device_placement并设置为True.}
\NormalTok{sess }\OperatorTok{=} \NormalTok{tf.Session(config}\OperatorTok{=}\NormalTok{tf.ConfigProto(log_device_placement}\OperatorTok{=}\VariableTok{True}\NormalTok{))}
\CommentTok{# 运行这个 op.}
\BuiltInTok{print} \NormalTok{sess.run(c)}
\end{Highlighting}
\end{Shaded}

你应该能看见以下输出:

\begin{verbatim}
Device mapping:
/job:localhost/replica:0/task:0/gpu:0 -> device: 0, name: Tesla K40c, pci bus
id: 0000:05:00.0
b: /job:localhost/replica:0/task:0/gpu:0
a: /job:localhost/replica:0/task:0/gpu:0
MatMul: /job:localhost/replica:0/task:0/gpu:0
[[ 22.  28.]
 [ 49.  64.]]
\end{verbatim}

\subsection{手工指派设备 }\label{ux624bux5de5ux6307ux6d3eux8bbeux5907}

如果你不想使用系统来为 operation 指派设备, 而是手工指派设备, 你可以用
\texttt{with\ tf.device} 创建一个设备环境, 这个环境下的 operation
都统一运行在环境指定的设备上.

\begin{Shaded}
\begin{Highlighting}[]
\CommentTok{# 新建一个graph.}
\ControlFlowTok{with} \NormalTok{tf.device(}\StringTok{'/cpu:0'}\NormalTok{):}
  \NormalTok{a }\OperatorTok{=} \NormalTok{tf.constant([}\FloatTok{1.0}\NormalTok{, }\FloatTok{2.0}\NormalTok{, }\FloatTok{3.0}\NormalTok{, }\FloatTok{4.0}\NormalTok{, }\FloatTok{5.0}\NormalTok{, }\FloatTok{6.0}\NormalTok{], shape}\OperatorTok{=}\NormalTok{[}\DecValTok{2}\NormalTok{, }\DecValTok{3}\NormalTok{], name}\OperatorTok{=}\StringTok{'a'}\NormalTok{)}
  \NormalTok{b }\OperatorTok{=} \NormalTok{tf.constant([}\FloatTok{1.0}\NormalTok{, }\FloatTok{2.0}\NormalTok{, }\FloatTok{3.0}\NormalTok{, }\FloatTok{4.0}\NormalTok{, }\FloatTok{5.0}\NormalTok{, }\FloatTok{6.0}\NormalTok{], shape}\OperatorTok{=}\NormalTok{[}\DecValTok{3}\NormalTok{, }\DecValTok{2}\NormalTok{], name}\OperatorTok{=}\StringTok{'b'}\NormalTok{)}
\NormalTok{c }\OperatorTok{=} \NormalTok{tf.matmul(a, b)}
\CommentTok{# 新建session with log_device_placement并设置为True.}
\NormalTok{sess }\OperatorTok{=} \NormalTok{tf.Session(config}\OperatorTok{=}\NormalTok{tf.ConfigProto(log_device_placement}\OperatorTok{=}\VariableTok{True}\NormalTok{))}
\CommentTok{# 运行这个op.}
\BuiltInTok{print} \NormalTok{sess.run(c)}
\end{Highlighting}
\end{Shaded}

你会发现现在 \texttt{a} 和 \texttt{b} 操作都被指派给了 \texttt{cpu:0}.

\begin{verbatim}
Device mapping:
/job:localhost/replica:0/task:0/gpu:0 -> device: 0, name: Tesla K40c, pci bus
id: 0000:05:00.0
b: /job:localhost/replica:0/task:0/cpu:0
a: /job:localhost/replica:0/task:0/cpu:0
MatMul: /job:localhost/replica:0/task:0/gpu:0
[[ 22.  28.]
 [ 49.  64.]]
\end{verbatim}

\subsection{在多GPU系统里使用单一GPU}\label{ux5728ux591agpuux7cfbux7edfux91ccux4f7fux7528ux5355ux4e00gpu}

如果你的系统里有多个 GPU, 那么 ID 最小的 GPU 会默认使用. 如果你想用别的
GPU, 可以用下面的方法显式的声明你的偏好:

\begin{Shaded}
\begin{Highlighting}[]
\CommentTok{# 新建一个 graph.}
\ControlFlowTok{with} \NormalTok{tf.device(}\StringTok{'/gpu:2'}\NormalTok{):}
  \NormalTok{a }\OperatorTok{=} \NormalTok{tf.constant([}\FloatTok{1.0}\NormalTok{, }\FloatTok{2.0}\NormalTok{, }\FloatTok{3.0}\NormalTok{, }\FloatTok{4.0}\NormalTok{, }\FloatTok{5.0}\NormalTok{, }\FloatTok{6.0}\NormalTok{], shape}\OperatorTok{=}\NormalTok{[}\DecValTok{2}\NormalTok{, }\DecValTok{3}\NormalTok{], name}\OperatorTok{=}\StringTok{'a'}\NormalTok{)}
  \NormalTok{b }\OperatorTok{=} \NormalTok{tf.constant([}\FloatTok{1.0}\NormalTok{, }\FloatTok{2.0}\NormalTok{, }\FloatTok{3.0}\NormalTok{, }\FloatTok{4.0}\NormalTok{, }\FloatTok{5.0}\NormalTok{, }\FloatTok{6.0}\NormalTok{], shape}\OperatorTok{=}\NormalTok{[}\DecValTok{3}\NormalTok{, }\DecValTok{2}\NormalTok{], name}\OperatorTok{=}\StringTok{'b'}\NormalTok{)}
  \NormalTok{c }\OperatorTok{=} \NormalTok{tf.matmul(a, b)}
\CommentTok{# 新建 session with log_device_placement 并设置为 True.}
\NormalTok{sess }\OperatorTok{=} \NormalTok{tf.Session(config}\OperatorTok{=}\NormalTok{tf.ConfigProto(log_device_placement}\OperatorTok{=}\VariableTok{True}\NormalTok{))}
\CommentTok{# 运行这个 op.}
\BuiltInTok{print} \NormalTok{sess.run(c)}
\end{Highlighting}
\end{Shaded}

如果你指定的设备不存在, 你会收到 \texttt{InvalidArgumentError} 错误提示:

\begin{verbatim}
InvalidArgumentError: Invalid argument: Cannot assign a device to node 'b':
Could not satisfy explicit device specification '/gpu:2'
   [[Node: b = Const[dtype=DT_FLOAT, value=Tensor<type: float shape: [3,2]
   values: 1 2 3...>, _device="/gpu:2"]()]]
\end{verbatim}

为了避免出现你指定的设备不存在这种情况, 你可以在创建的 \texttt{session}
里把参数 \texttt{allow\_soft\_placement} 设置为 \texttt{True}, 这样
tensorFlow 会自动选择一个存在并且支持的设备来运行 operation.

\begin{Shaded}
\begin{Highlighting}[]
\CommentTok{# 新建一个 graph.}
\ControlFlowTok{with} \NormalTok{tf.device(}\StringTok{'/gpu:2'}\NormalTok{):}
  \NormalTok{a }\OperatorTok{=} \NormalTok{tf.constant([}\FloatTok{1.0}\NormalTok{, }\FloatTok{2.0}\NormalTok{, }\FloatTok{3.0}\NormalTok{, }\FloatTok{4.0}\NormalTok{, }\FloatTok{5.0}\NormalTok{, }\FloatTok{6.0}\NormalTok{], shape}\OperatorTok{=}\NormalTok{[}\DecValTok{2}\NormalTok{, }\DecValTok{3}\NormalTok{], name}\OperatorTok{=}\StringTok{'a'}\NormalTok{)}
  \NormalTok{b }\OperatorTok{=} \NormalTok{tf.constant([}\FloatTok{1.0}\NormalTok{, }\FloatTok{2.0}\NormalTok{, }\FloatTok{3.0}\NormalTok{, }\FloatTok{4.0}\NormalTok{, }\FloatTok{5.0}\NormalTok{, }\FloatTok{6.0}\NormalTok{], shape}\OperatorTok{=}\NormalTok{[}\DecValTok{3}\NormalTok{, }\DecValTok{2}\NormalTok{], name}\OperatorTok{=}\StringTok{'b'}\NormalTok{)}
  \NormalTok{c }\OperatorTok{=} \NormalTok{tf.matmul(a, b)}
\CommentTok{# 新建 session with log_device_placement 并设置为 True.}
\NormalTok{sess }\OperatorTok{=} \NormalTok{tf.Session(config}\OperatorTok{=}\NormalTok{tf.ConfigProto(}
      \NormalTok{allow_soft_placement}\OperatorTok{=}\VariableTok{True}\NormalTok{, log_device_placement}\OperatorTok{=}\VariableTok{True}\NormalTok{))}
\CommentTok{# 运行这个 op.}
\BuiltInTok{print} \NormalTok{sess.run(c)}
\end{Highlighting}
\end{Shaded}

\subsection{使用多个 GPU }\label{ux4f7fux7528ux591aux4e2a-gpu}

如果你想让 TensorFlow 在多个 GPU 上运行, 你可以建立 multi-tower 结构,
在这个结构 里每个 tower 分别被指配给不同的 GPU 运行. 比如:

\begin{verbatim}
# 新建一个 graph.
c = []
for d in ['/gpu:2', '/gpu:3']:
  with tf.device(d):
    a = tf.constant([1.0, 2.0, 3.0, 4.0, 5.0, 6.0], shape=[2, 3])
    b = tf.constant([1.0, 2.0, 3.0, 4.0, 5.0, 6.0], shape=[3, 2])
    c.append(tf.matmul(a, b))
with tf.device('/cpu:0'):
  sum = tf.add_n(c)
# 新建session with log_device_placement并设置为True.
sess = tf.Session(config=tf.ConfigProto(log_device_placement=True))
# 运行这个op.
print sess.run(sum)
\end{verbatim}

你会看到如下输出:

\begin{verbatim}
Device mapping:
/job:localhost/replica:0/task:0/gpu:0 -> device: 0, name: Tesla K20m, pci bus
id: 0000:02:00.0
/job:localhost/replica:0/task:0/gpu:1 -> device: 1, name: Tesla K20m, pci bus
id: 0000:03:00.0
/job:localhost/replica:0/task:0/gpu:2 -> device: 2, name: Tesla K20m, pci bus
id: 0000:83:00.0
/job:localhost/replica:0/task:0/gpu:3 -> device: 3, name: Tesla K20m, pci bus
id: 0000:84:00.0
Const_3: /job:localhost/replica:0/task:0/gpu:3
Const_2: /job:localhost/replica:0/task:0/gpu:3
MatMul_1: /job:localhost/replica:0/task:0/gpu:3
Const_1: /job:localhost/replica:0/task:0/gpu:2
Const: /job:localhost/replica:0/task:0/gpu:2
MatMul: /job:localhost/replica:0/task:0/gpu:2
AddN: /job:localhost/replica:0/task:0/cpu:0
[[  44.   56.]
 [  98.  128.]]
\end{verbatim}

\href{tensorflow-zh/SOURCE/tutorials/deep_cnn/index.md}{cifar10
tutorial} 这个例子很好的演示了怎样用GPU集群训练.

\begin{quote}
原文:\href{http://tensorflow.org/how_tos/using_gpu/index.md}{using\_gpu}
翻译:{[}@lianghyv{]}(https://github.com/lianghyv)
校对:\href{https://github.com/jikexueyuanwiki}{Wiki}
\end{quote}

